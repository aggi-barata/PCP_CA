\section{Hardness Results}\label{section:hardness}

We begin by borrowing some machinery from the PCP literature. 
We will deal with Boolean functions $A:\mathbb{F}^u_2 \rightarrow
\{-1, 1\}$.  A function is called linear if $A(x \oplus y) = A(x)
\cdot A(y)$, where $\oplus$ denotes vector addition over
$\mathbb{F}_2$. Note that there are precisely $2^u$ linear functions:
for every $\alpha\ \in \ \mathbb{F}^u_2$, $\chi_\alpha$
defined as
\[
\chi_\alpha(a) = (-1)^{a \cdot \alpha} ~~\forall a \in \mathbb{F}^u_2
\]


\paragraph{Hadamard Codes.} Our PCP verifier will
expect an encoding of the proof described in the prequel.
Specifically, we will use Hadamard codes which we define now.


\begin{definition}[Hadamard Encoding]
Hadamard code of $p \in \mathbb{F}^{u}_2$ is the $2^u$¯bit string 
$\{\chi_p(a)\}_{a \in \mathbb{F}^u_2}$. We denote it by Hadamard($p$).
\end{definition}

Recall that the string $x = Q(W)$ read by the aforementioned verifier
is supposed to satisfy certain linear conditions modulo $2$. Let these
constraints be $h_1 \cdot x = \eta_1$, \ldots $h_u \cot x = \eta_u$,
where $h_1, h_2, h_3 \ldots h_u \in \mathbb{F}^{3u}_2$ and $\eta_1,
\eta_2 \ldots \eta_u \in \mathbb{F}_2$.

\paragraph{Folding.}Informally, {\em Folding} is a technique used to enable the verifier to 
ignore the linear constraint test. Suppose that $C$ is a Hadamard code
of $x$ and $x$ satisfies the constraints mentioned above.  Let
${\mathcal H}$ be the linear subspace spanned by the vectors $h_1, h_2
\ldots h_u$.  Then for any vector $b$ and any vector $h \in {\mathcal
  H}$, $h = \oplus \rho_i \cdot h_i$, we have
\[
      C(b \oplus h) = C(b)\cdot(-1)^{\sum_i \rho_i \eta_i}
\]

One may generalize this observation, for any function $F:
\mathbb{F}^{3u}_2 \rightarrow \{-1, 1\}$, Let $v_b$ denote the
lexicographically smallest vector in the set of vectors $b \oplus
{\mathcal H}$ (coset of ${\mathcal H}$).  we one may define a function
$F'$ as:
\[
b = v_b \oplus \rho_i\cdot h_i, ~~ \rho_1,\ldots \rho_u \in \mathbb{F}_2
F'(b) = F(b) \cdot (-1)^{\sum_i \rho_i \eta_i}
\]
$F'$ is said to be a folding of $F$ over the linear constraints.

The verifier we design requires the supposed Hadamard codes be folded
over the respective constraints. This does not alter much. One
difficulty that it may introduce is ensuring that this requirement is
enforced.  This is done as follows: If the verifier is required to
read $F(b)$ from the code, it can read $F(v_b)$ and can calculate
$F(b)$ from the expression given in the prequel.

A comment on folding of Hadamard codes is due. We will eventually need
to show that verifier accepts the purported proofs with a good enough
probability, then these ``supposed'' proofs can be decoded to
construct an assignment of the variables which satisfies a significant
fraction of the constraints. Decoding of $F$ gives $h$ with
probability $\widehat{H}^2_h$. Now, {\em folding} ensures that any $h$
given by this decoding procedure satisfies the linear constraints on
variables. Thus, folding lets us to ignore the linear constraints 
altogether.

\subsection{The PCP verifier}
We now define and analyze our PCP verifier whose test
s will be linear equations over $3$ 
variables. The verifier expects to have
proofs $(\Pi_A, \Pi_B)$ which are the Hadamard encoding of the assignments
given to the vertices in $A$ and $B$. The verifier does the following:
\begin{enumerate}
\item Pick at random edge $e = (x,y)$. Let $f$, $g$ be the corresponding
 Hadamard encoding of the labels of vertices $x$ and $y$ respectively; 
  and $h$ be the corresponding projection function between $\Sigma_A$ and $\Sigma_B$.
$h^{-1}(q)$ denotes the unique vector $p \in \Sigma_A$ such that $h(p)  = q$. 

\item Choose ${\bf v,y}$ at random conditioned on ${\bf v} \in \{\pm 1\}^{\log |\Sigma_A|}$ 
and ${\bf y} \in \{\pm 1\}^{\log |\Sigma_B|}$. Accept iff 
\[
           f({\bf v}) \cdot g({\bf y}) = f(h^{-1}({\bf y}) \oplus {\bf v})
\]
\end{enumerate}

We begin by stating a Theorem of Khot \cite{Khot01}. Khot uses a
Hadamard based PCP on an instance of {\sc Label-cover} produced by
applying parallel repetition on a linear PCP with completeness $1 -
\epsilon$ and soundness $\mu$. The linear PCP is queried $k$ times in
parallel and has a soundness of $S$. We start with a {\sc Label-Cover} with 
better parameters, so we set $k = 1$. We now take the liberity and rephrase 
Khot's Theorem about the guarantees provided by the verifier $V_{lin}$ 
(Page 4, Theorem 3.1 in \cite{Khot01}) in the language of {\sc Label-Cover}.


\begin{theorem} \label{khot}
Every {\sc Label-Cover} instance with linear projection tests and size $n$
has a verifier $V_{lin}$ which
\begin{itemize}

\item Uses $r = \log n$ + ${\cal O}(1)$ random bits.

\item Performs linear tests each with arity $3$.

\item Has completeness at least $1 - \epsilon$.

\item Has soundness $\frac{1}{2} + \delta$ provided $ S < \delta^2$,
  where $S$ is the soundness of {\sc Label-Cover} instance.
\end{itemize}
\end{theorem}

\begin{theorem}\label{3lin}
 For some $\alpha, \beta > 0$, {\sc 3SAT} has a verifier ${\cal V}$ that has the following properties.
\begin{itemize}
\item Uses $\log n \cdot \left(1 + o\left(1\right)\right)$ random bits.

\item The tests performed by ${\cal V}$ are linear over $3$ variables. 

\item If the {\sc SAT} instance is satisfiable, ${\cal V}$ accepts it with probability at 
least $1 - \frac{1}{(\log n)^\alpha}$.

\item If it is unsatisfiable, ${\cal V}$ accepts any proof with probability at most $\frac{1}{2} + \frac{1}{(\log n)^{\beta'}}$.
\end{itemize}

\end{theorem}
\noindent{\em Proof.}  We now use Khot's verifier $V_{lin}$ on the
{\sc Label-Cover} instance in Corollary \ref{oldcover}. Specifically,
set $\epsilon = \frac{1}{(\log n)^\alpha}$, and $\delta = \frac{1}{(\log
n)^{\sqrt[3]{\beta}}}$.  Completeness follows trivially from Theorem
\ref{khot}. Since $S < \delta^2$, the soundness of $V_{lin}$ is $\frac{1}{2} +
\frac{1}{(\log n)^{\beta'}}$. This completes the proof. \qed


%{\em Completeness:} The verifier's test has no
%additional error apart from that introduced by the outer (Theorem
%\ref{sound}). In the good case, at least $1 - 1/N^k$ fraction of the
%edges can be satisfied. So, the probability of the verifier accepting
%the label-cover instance is $1 - 1/N^k$.\\
%{\em Soundness:}  Follow from Lemma \ref{soundness} that blah...blah.

\subsection{Inapproximability Results}

The PCP theorem \ref{3lin} immediately yields a hardness of $\frac{1}{2} +
\frac{1}{\log n)^\beta}$ for {\sc 3Lin} assuming ${\sf P \ne NP}$. This
translates into a hardness factor of
$\frac{7}{8} + \frac{1}{(\log n)^\beta}$ for {\sc 3Sat}.\\

\begin{corollary}[{\sc 3Lin Hardness}]
  For some $\alpha, \beta > 0$, given a {\sc 3Lin} instance $\Phi$ of
  size $n$ it is {\sf NP}-hard to distinguish between the following
  two cases.
\begin{itemize}
\item There is an assignment which satisfies $(1 - \frac{1}{(\log n)^\alpha})$-fraction of $\Phi$.
\item Every assignment can satisfy at most $(\frac{1}{2} + \frac{1}{(\log
      n)^{\beta}})$-fraction of $\Phi$.
\end{itemize}
\end{corollary}
\noindent {\em Proof.} The constraint satisfaction version of Theorem
\ref{3lin}. \qed \\

\begin{corollary}[{\sc 3Sat Hardness}]
  For some $\alpha, \beta > 0$, it is {\sf NP}-hard to distinguish if a given 
{\sc 3Sat} instance $\Upsilon$  of size $n$ which of the following holds.
\begin{itemize}
\item There is an assignment which satisfies $(1 -
    \frac{1}{(\log n)^\alpha})$-fraction of $\Upsilon$.
\item Every assignment can satisfy at most $(\frac{7}{8} + \frac{1}{(\log n)^{\beta}})$-fraction of $\Upsilon$.
\end{itemize}
\end{corollary}

