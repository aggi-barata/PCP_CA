\section{Polynomially Small Completeness Error}\label{section:Iterated}

In this section, we construct PCPs with polynomially low completeness
error and yet, have a low (even sub-constant) soundness error. The
bottleneck to achieving polynomially low completeness error was that
the number of queries grew hand in hand with the completeness
error. Thus, if we set $r = \log \log n$ in Lemma \ref{Camplify} to
get polynomially small completeness error, the queries made will be
{\sf Poly}($n$). At which point, any technique to rescue the query
size will result in either perturbing the completeness error to a
constant or soundness error into the $1/{\sf polylog}$ regime. And
both these means defeat the ends that we set out to achieve as the
former the destroys the completeness error. While the latter shall
take us into a region which we currently cannot come out of. Thus,
both these approaches are ruled out.

However, we cannot allow the query size to grow along the completeness
if we are to achieve out ends. The approach we adopt is simple, we use
composition to reduce the query size and at this point of time, there
are a lot of already established composition theorems. However, none
of them will be suitable to our needs as we need a low completeness
error and almost all the exant theorems either are non-linear or have
a constant error in completeness. So, in a sense we are forced to
self-compose with our constructions. And as it turns out it is not the
worst thing that can happen. To that end, we start by recollecting
Corollary \ref{iterate} that we construct after a tensoring and
performing sequential repetition on a basic assignment tester obtained
from Lemma \ref{Tester}.

\begin{col}
  For every $\rho > 0$ and $\delta < 1/4$, there is an absolute constant
  $\vartheta$, such that an ($1 - \rho, \delta$)-assignment tester
  that uses $r_v$ random bits to make $q$ queries, we can construct an
  $(1 - {\cal O}\left(\rho^2\right) , {\delta})$-assignment
  tester. The new assginment tester uses $2 \cdot r_v + {\cal
    O}(\vartheta)$ random bits and makes ${\cal O}(\vartheta \cdot
  q^2)$ queries. Moreover, our latest construction has a linear
  predicate.
 \end{col}



 \noindent Let us take a moment and reflect the hypothetical
 parameters that one might obtain if we composed the above Corollary
 with itself (presented in Table \ref{table:selfcompose}). In short,
 we would double the error while reducing the queries as a function of
 $n$ (ref. Claim \ref{dying})\footnote{We prove a strong version of
   this Claim at a later point}. So, our blue print of the
 construction will be as follows.

 \begin{table}[h]
\centering 
\begin{tabular}{|l|c|c|}
  \hline 
  \ & & \\
  {\em Parameters}  &{\sc Inner} & \parbox{1.65in} {\sc Self Composed Inner}  \\
  \hline
%\ &  & & \\ 
  \hline
  {\tt Completeness error} & $\rho(n)$ & $2 \cdot \delta(n)$  \\  
  {\tt Soundness error} &  $\delta(n)$ & $2 \cdot \delta(n)$   \\
  {\tt Queries} &   $q(n)$ & $\q(\q(n))$  \\
  {\tt Randomness}&  $r(n)$ & $r(n) + r(r(n))$  \\ 
  {\tt Alphabet} & $\{0,1\}$  & $\{0,1\}$ \\
\hline
\end{tabular} %\caption{Parameters after one iteration} \label{1round}
 \caption{Parameters after Self-Composition of Inner} \label{table:selfcompose}
\end{table}



\begin{claim}\label{dying}
  For every non-decreasing $f : \mathbb{N} \rightarrow \mathbb{N}$
  such that $f(n) = o(n)$, the following holds for every $x$.
\begin{itemize}
\item $f\left(f\left(x\right)\right) \le f(x)$.
\item $f\left(f\left(x\right)\right) <  x/k$, for all $k > 0$.
\end{itemize}
\end{claim}
\noindent {\em Proof.} The first part of the claim follows from the
hypothesis that $f$ is a non-decreasing, sub-linear function. For the
latter half, suppose that $f\left(f\left(x\right)\right) = x/k$, for
some constant $k$. This implies $f(x) = O(n)$ and this is a
contradiction. Thus, the claim holds.\qed



\subsection*{Modified Composition}\label{modified}
\begin{enumerate}
\item We start with the basic $(1 - \rho,\delta)$-assignment tester
  constructed in Lemma \ref{Tester}. We then tensor the ($1 -
  \rho,\delta$)-assignment tester to obtain a ($1 -
  \rho^2,\delta^2$)-tester. As in Lemma \ref{boosting} we perform
  soundness amplification in a randomness efficient manner to obtain
  $(\vartheta \cdot \rho^2, \delta)$-assignment tester, denoted by
  ${\cal V}$. The query complexity $\q(n)$ is $\vartheta \cdot \q^2$,
  where $\q$ is the number of queries made by the assignment tester in
  Lemma \ref{Tester}.

\item We now use self-composition to compose ${\cal V}$ with
  itself. As we have witnessed in Table \ref{table:selfcompose}, this
  doubles the error parameters\footnote{Will be established via Lemma
    \ref{robust-composition}}. Thus, we end up with a ($ 1 - 2 \cdot
  \vartheta \cdot \rho^2, 2 \cdot \delta$)-tester with a query
  complexity of $\q(\q(n))$, where $n$ is the size of the outer.  Set
  $\vartheta = 5 , \rho = 1/( 8 \cdot \vartheta), \delta = 16/100$ to
  check that we essentially have a $(1 - {\cal O}(\rho^2),
  \delta$)-tester
  
\item We then iterate over the first two steps $ \simeq \log n$ times.
  It follows that the tester we achieve after this is ($ 1 - 1/{\cal
    O}(\rho^{\log n}), \delta'$)-tester. This is what we would like to
  achieve as ${\cal O}(\rho^{\log n}) \simeq {\cal
    O}(1/n^{\Omega(1)})$. The query complexity of our construction
  decreases over each iteration in $n$. It take a bit of work but it
  is not hard to show that at the end of it all, we end up with a
  tester that only makes a constant number of queries.
\end{enumerate}

However, all is not well as the above sketch suggests. For one, we
have designed the basic tester to be utilized as an inner. To be
eligible for composition as an outer, we need an robust variant of the
same. We shall use this opportunity to refresh the notion of verifier
composition. The prover would need to provide an inner-proof
$\psi_{R}$ for every possible random coin $R$ of the outer PCP. Thus,
the proof for the composed verifier is $\Psi = \big\{ \psi_{R}\ :\ R \
\mbox{outer random coins} \big\}$. Thus, if our tester to be eligible
to be used as an outer, it needs to be robust. Informally, any
satisfying assignment must be far from an unsatisfying assignment.  We
now formalize this notion via {\em robust} assignment testers first
introduced by \cite{DR,BGHSV}.

\begin{definition}[Robust Assignment Testers]\label{RAT}
  An assignment-tester is called $\rho$-robust if in the soundness
  case in Definition \ref{AT} of an assignment tester, for every
  assignment $\tilde{\pi} : Y \rightarrow \Sigma$, the assignment
  $(\pi \cup \tilde{\pi})|_{\psi_i}$ is $\rho$-far from any satisfying
  assignment of $\psi_i$ on least $1 - \epsilon$ fraction of $\Psi =
  \big \{\psi_1, . . . , \psi_R \big\}$.
\end{definition}

Dinur and Reingold \cite{DR} provide a generic transformation that
transforms any assignment-tester into a robust one. In particular,
they prove the Robustization lemma which is very useful in our
quest. Specifically, we use a slight variant of their result -- one
that includes completeness error. Let us denote a tester that uses $R$
random bits, has a completeness error $\rho$, size $S$, makes $\q$
queries, distance parameter $\delta$, soundness $\epsilon$ and
robustness parameter $\mu$ by $(R, \rho, S, \q, \delta, \epsilon,
\mu)$.

\begin{lemma}[Robustization Lemma, Lemma 3.6 in \cite{DR}]\label{Generic}
  There exists some $c_1 > 0$ such that given an assignment tester
  ${\cal A}$ with parameters $(R, \rho, S, \q, \delta, \epsilon)$, we
  can construct a $\rho$-robust assignment tester ${\cal A}' = \Psi'$
  with parameters $\left(R, \rho, c_1 \cdot S, \q, \delta,\epsilon,
    \mu = \Omega(1/q)\right)$.
\end{lemma}

\noindent Based on Lemma \ref{Generic}, we can now transform our
tester obtained from Lemma \ref{Tester} into a robust tester. The key
parameter for us is the completeness error and this is left untouched
by this generic robustization. We also recollect the robust
composition lemma from \cite{DR} which basically establishes that
robust assignment-testers compose in a modular fashion.

\begin{lemma}[Robust Composition Lemma 3.5 in
  \cite{DR}]\label{robust-composition}
  Let ${\cal A}_1$ and ${\cal A}_2$ be two assignment testers with
  parameters $(R_1, \rho_1, \s_1, \q_1, \delta_1, \epsilon_1)$ and $(R_2,
  \rho_2, \s_2, \q_2, \delta_2, \epsilon_2)$ respectively. If ${\cal A}_1$
  is $\mu$-robust with $\mu = \delta_2$ then one can construct an
  assignment tester ${\cal A}_3$ with parameters $(R_3, \rho_3, \s_3, \q_3,
  \delta_3, \epsilon_3)$ such that:
\[
R_3 = R_1(n) \cdot R_2(\s_1(n)), \ \ \ \s_3(n) = \s_1(\s_2(n)), \ \ \
\q_3(n) = \q_2(\s_1(n)), \ \ \rho_3 = \rho_1(n) + \rho_2(\s_1(n)) - \rho_1(n) \cdot
\rho_2(\s_1(n))\\
\]
and
\[
\epsilon_3 = \epsilon_1(n) + \epsilon_2(\s_1(n)) - \epsilon_1(n) \cdot \epsilon_2(\s_1(n)), \ \ \ \delta_3(n) = \delta_1(n)
\]
\end{lemma}


\subsection{Construction of Tester}

We now highlight the iterative construction aimed at constructing
PCP's with polynomially small completeness error. The construction
involves {\em three} phases as outlined in Section \ref{modified}.\\

\noindent {\sf Tensoring and Soundness Amplification.} 
We begin by recalling the tester we obtained after tensoring a
($1 - \rho, \delta$)-tester followed by $\vartheta$ sequential repetitions in a
randomness efficient manner via Corollary \ref{iterate}.

\begin{col}[Tensoring + Repetitions]
 For every $\rho > 0, \delta < 1/4$, there is an absolute constant
  $\vartheta$, such that an ($1 - \rho, \delta$)-assignment tester
  that uses $r_v$ random bits to make $q$ queries, we can construct an
  $(1 - {\cal O}\left(\rho^2\right) , {\delta})$-assignment
  tester. The new assginment tester uses $2 \cdot r_v + {\cal
    O}(\vartheta)$ random bits and makes ${\cal O}(\vartheta \cdot
  q^2)$ queries. Moreover, our latest construction has a linear
  predicate.
\end{col}

\noindent {\sf Robustization.} In order to compose the tester we have
built with itself.  It must be robust in the sense of Definition
\ref{RAT}. So, our approach will be to robustize the tester via Lemma
\ref{Generic} and then compose it with the tester obtained by
Corollary \ref{iterate} using Lemma \ref{robust-composition}.

\begin{lemma}[Robust Tester]\label{robustTester}
  There exists some $\gamma > 0, \delta < 1/4$, such that as ($1 -
  \rho, \delta$)-tester ${\cal T}$ that uses $r_v$ random bits to make
  $q$ queries to verify a Boolean property $\psi$ can be made a
  $\mu$-robust assignment tester ${\cal A}$ with parameters
  $\left(r_v, \rho, \gamma \cdot \s({\cal T}), \q, \delta, \epsilon,
    \mu = \Omega(\frac{1}{\q})\right)$.
\end{lemma}
\noindent {\em Proof.} We invoke Lemma \ref{Generic} on the
($\rho,\delta$)-tester from Lemma \ref{Tester}. \qed \\ 

\noindent {\sf Self Composition.} We now compose the robust tester
obtained by Lemma \ref{robustTester} with the tester obtained via
Corollary \ref{iterate} as its inner.

\begin{lemma}\label{selfcomposed}
  For every $\rho > 0$ and $\delta < 1/4$, there is some $\epsilon > 0$,
  the composition of the assignment tester obtained via Corollary \ref{iterate} with
 itself in a robust way yields an assignment tester with the following 
parameters (${\cal O}(r_v)$, ${\cal O}(\rho^2)$, $\s(\s(n))$, $\q(\q(n))$, $\delta$, $\epsilon$)-tester.
\end{lemma}
\noindent {\em Proof.} The idea is very simple. We start with the basic tester
($r_v$, $\rho^2$, $\s(n)$, $\q(n)$, $\delta$, $s'$)-tester obtained from Corollary \ref{iterate}. 
We then invoke Lemma \ref{Generic} on it to obtain its robust cousin 
$\left(r_v, \rho^2, \gamma \cdot \s(n), \q(n), \delta, \epsilon, \mu =
  \Omega(1/\q(n))\right)$-tester. We now compose them in an obvious 
way using Lemma \ref{robust-composition} to claim the result. 

In particular, we begin by setting $\vartheta = 5, \rho = 1/( 8 \cdot \vartheta)$ in Lemma \ref{iterate} to
obtain an ($1/64 \cdot \vartheta, \delta'$)-tester with $\q$ queries.
Upon composition with a robust version of itself we end up with a $\rho = 1/32
\cdot \vartheta = 1/2 \cdot 1/64 \cdot \vartheta \simeq {\cal O}(\rho^2)$
and $\delta = \Omega(\frac{1}{\q}) = {\cal O}(1) \simeq \epsilon$\footnote{Since, this is a
  one-time composition, it is a constant.}. The rest of
the parameters follow vacuously.\qed 

\subsection*{Iterated Tester}

We are now almost set to iterate over Corollary \ref{iterate}, Lemmas
\ref{robustTester} and \ref{selfcomposed} to obtain our ends. But
before we go any further, we would like to introduce a couple of
definitions and make a few observations about the function
${\q}(n)$. At the very outset, we make it clear that in what follows
we deal with only sub-linear functions. That is, functions whose
output is small than its input by a {\sf poly}($\cdot$)
quantity. Examples of such functions include $\log n$, {\sf poly}
$\log n$, $n^\epsilon$, for $\epsilon < 1$.

\begin{definition}[{\tt Repetition}] \label{repetition} We define {\tt
    Repetition}(${\cal V}$) as an invocation of Corollary \ref{iterate} on
  ${\cal V}$, followed by robustization using Lemma
  \ref{robustTester}, and finally self composition using Lemma
  \ref{robust-composition}. Further, for any $k > 0$ we define {\tt
    Repetition}$^{k}$(${\cal V}$) recursively as {\tt Repetition}({\tt
    Repetition}$^{k-1}$(${\cal V}$)).
\end{definition}


\begin{corollary}\label{repeat} For every $\rho > 0$ and $\delta < 1/4$, there is a  $k > 0$ such that
  {\tt Repetition}(${\cal T}$) yields a ($k \cdot r_v, {\cal O}(\rho^2), {\cal O}(\s^k), \delta$)-tester.
\end{corollary}
\noindent {\em Proof.}  Essentially restating Lemma \ref{selfcomposed}
expect for we use an explicit constant to account for the randomness
used by in a single invocation of {\tt Repetition}. \qed

\begin{definition}\label{queries}
  Let ${\cal T}$ be an assignment tester with a query function $\q :
  \mathbb{N} \rightarrow \mathbb{N}$.  For any $i$, we define $\q^{i}(n)$
    as the query size of the tester produced after {\tt Repetition}$^i({\cal T})$.
\end{definition}

\begin{proposition}\label{compute}
  For every tester ${\cal T}$ with query complexity $\q(n)$, $k > 0$,
  $\q^{k}(n) = \big[\q^{k-1}\left((\q^{k-1}(n))^2\right)\big]^2$.
\end{proposition}
\noindent {\em Proof.}  Follows from the definition of {\tt
  Repetition}. The query size of the input tester to the {\tt
  Repetition}$^i$ is $\q^{i-1}$. Recall Corollary \ref{iterate} to conclude
that the tensor and sequential repetition operations increase the
query function to $Q = {\cal O}(\q^{i-1}(n))^2$. The robustization
step does not increase the number of queries. And from Lemma
\ref{robust-composition}, the self composition step results in a query
size of $Q(Q(n))$, which is
$\big[\q^{k-1}\left((\q^{k-1}(n))^2\right)\big]^2$. \qed
 
\begin{lemma}
  The sequence ${\cal Q}_{i \in \mathbb{N}} = \{\q^{i}(n)\}$ is well
  behaved in the sense that given $\q^k$, one can compute $\q^{k-1}$
  and $\q^{k+1}$ upto constant factors.
\end{lemma}
\noindent {\em Proof.} In order to be able to compute $\q^{k-1}$ we
would need uncompute the effect of the $k$-th {\tt Repetition} and we
are perfectly capable to do $\grave{a}$ la Proposition \ref{compute}. The
argument for computing $\q^{k+1}$ follows arguments from Proposition
\ref{compute}. \qed


\begin{corollary}[Smoothness Property]\label{smooth}
  The sequence ${\cal Q}_{i \in \mathbb{N}} = \{\q^{i}(n)\}$ is smooth
  in the following sense. 
\[
        \q^{k} = \Theta(f) \implies \q^{k-1} = \Theta(\tilde{f}) \quad \mbox{and} \quad \q^{k+1} = \Theta(\hat{f})
\]
Moreover, both $\tilde{f}$ and $\hat{f}$ are sub-linear in their input.
\end{corollary}
%\noindent {\em Proof.} .  \qed


\begin{definition}[${\cal Q} = \Upsilon(f)$]
  For any $f : \mathbb{N} \rightarrow \mathbb{N}$ and any ${\cal Q}_{i
    \in \mathbb{N}} = \{\q^{i}(n)\}$, we say ${\cal Q} = \Upsilon(f)$
  if there is a constant $C$, such that $\q^{i}(n) = \Theta(f^i(n))$
  for every $i > C$, where $f^i(n)$ is the function obtained by
  composing $f$ with itself for $i$ times.
\end{definition}


\begin{lemma}\label{qdecay}
  The sequence ${\cal Q}_{i \in \mathbb{N}} = \{\q^{i}(n)\}$ decreases
  faster than any function in $n$. In other words, there is no $f$
  such ${\cal Q} = \Upsilon(f)$.
\end{lemma}
\noindent {\em Proof.} We give a proof by contradiction. Suppose there
is a function $f$ such that ${\cal Q} = \Upsilon(f)$. Thus for every
$i > C$, $\q^{i}(n) = \Theta(f^i(n))$. Since, every element in ${\cal
  Q}$ can be used to good effect in computing its predecessor, we can
go backwards until the point where we have reached the start of the
sequence or $\q^j$ is no longer sub-linear. In the former, we arrive
at a contradiction as the initial tester in the sequence had a query
complexity independent of $n$ or a constant number of queries. In the
latter case, we contradict the smoothness property of Corollary
\ref{smooth} as $\q^j = \Theta({f})$ implies the existence
of $\q^{j-1} = \Theta(\hat{f})$. \qed

\begin{proposition}\label{constant}
  ${\cal Q}_{i \in \mathbb{N}} = \{\q^{i}(n)\}$ is independent of
  $n$. In other words, any $\q^{i}(n) = {\cal O}(1)$.
\end{proposition}
\noindent {\em Proof.} It follows from Lemma \ref{qdecay} that the
sequence ${\cal Q}$ is not bounded from below by any $f(n)$. Thus,
${\cal Q}$ is independent of $n$. \qed

\begin{theorem}\label{iterateInner}
  For every $k , \rho > 0$ and  $\delta < 1/4$, there exists $m = k(\cdot)$ and
  constants $\q', \epsilon > 0$, such that {\tt
    Repetition}$^m$($r_v, \rho, \s, \q, \delta$)-assignment tester yields a ($r_v
  \cdot \log k, \rho^k, \s^k, \q', \delta, \epsilon$)-tester. Also, the newly constructed assignment
 tester performs only linear tests.
\end{theorem}
\noindent {\em Proof.} Recall that even though sequential repetition,
self composition increase the size, the tensoring step dominates the
rest of them asymptotically. Since in this study, we primarily deal
with asymptotics, we can focus on the blow-up due to
tensoring. Repeated tensoring essentially squares the size of the
input instance. It is easy to verify that the sequence generated is
$\{2^{2^{i-1}}\}_{i}$.

Corollary \ref{repeat} gives the parameters after a single invocation
of {\tt Repetition}. To establish the Theorem would essentially
require us to export each parameter across each invocation of {\tt
  Repetition}. We invoke it with $m = \log \log k$. We now analyze
and argue the parameters of the tester.
\begin{itemize}

\item {\em Randomness:} As highlighted in the prequel, the randomness
  used is asympotically dominated by the tensoring operation. Thus, we
  would square the randomness used at every iteration. Hence if the
  input tester uses $r_v$ bits, after $m$ iterations the randomness
  blows up to $r_v \cdot 2^{2^{m}} \simeq r_v \cdot k$.

\item {\em Completeness:} Again, we square the completeness error in
  each iteration, the error falls doubly exponential in $m$, which is
  {\cal O}($k$). Hence, the completeness error at the end of $m$
  iterations is ${\cal O}(\rho^k)$.

\item {\em Query Size:} We recall Proposition \ref{constant} to
  conclude that the query $\q(n)$ is independent of $n$ and thus, we
  maintain a constant query size after each invocation.

\item{\em Soundness:} Say that {\tt Repetition}$^m$(($r_v, \rho, \s, \q, \delta$)-tester) is given access 
to a purported proof $\Pi$,  we need to show that the if the proof $\Pi_{\sigma}$ (denotes
 the projection of $\Pi$ on the variables of $\psi$) is $\delta'$-far to a satisfying proof of 
the Boolean predicate, we reject with probability at least $\Omega(\delta)$. We show this 
property holds after each iteration and hence, the newly constructed assignment tester via this 
process always has the soundness condition. We begin by defining the projection 
of a proof on $\sigma$.  We denote by $\Pi_\sigma$ the assignment which $\Pi$ makes to the variables in
$\sigma$. We begin by recalling Observation \ref{invariant}, which states that the distance of assignment to 
$\sigma$ is invariant to tensoring of proof. In fact, the same is vacously true of sequential repetition
as it does not modify the proof. We are now ready to establish the soundness.

The approach is simple. We give a proof by induction on the number of iterations. The assignment 
tester produced after the first iterate is sound by Corollary \ref{iterate}. Assume that the assignment 
tester produced after $i$-th iterations is sound. We now show that the assignment tester produced after ($i+1$)-th iteration is also sound. 
Since, the assignment tester at the $i$-th level has the soundness property, it follows 
from Corollary \ref{iterate} that the soundness holds even after $(i+1)$-th iteration. Thus, it is true
that the output of {\tt Repetition}$^m$(($r_v, \rho, \s, \q, \delta$)-tester)  is sound.



\item {\em Distance Parameter:} In our construction, this is same as
  the soundness.
\end{itemize} 
\noindent The linearity of the tests follows from the observation that
that the basic steps of this operation including tensoring, sequential
repetition, self composition preserve linearity.  \qed


\subsection{Final Composition}

\begin{theorem}\label{constantSoundness}
  For every $n$ and $\alpha > 0$ and some $\epsilon$,
\[
3SAT \in PCP_{1 - {n^{\frac{\alpha'}{\alpha' + 1}}}, {\epsilon}} \big[ \left( 1 + \alpha \right) \cdot \log n, {\cal O}(1) \big ]_{\{0,1\}}
\]
Moreover, verifier's test are linear, and also have the projection
property.
\end{theorem}
\noindent {\em Proof.}  We invoke Theorem
\ref{constantSoundness} to construct a $(1/ n^\alpha,
  \epsilon)$ PCP. We then compose the inner with the
Label-Cover instance in an obvious way. We analyze the parameters of
the newly composed verifier.

\begin{itemize}
\item Completeness: Since the Label-Cover has perfect completeness,
  the error is solely contributed by the inner. Invoking Theorem
  \ref{iterateInner}, we conclude that the completeness is $1 - n^{\frac{-\alpha'}{\alpha' + 1}}$.

\item Soundness: The inner and the outer each contribute $1/\epsilon$
  to the error. Hence, the overall soundness is $1/{\cal
    O}\left(\epsilon\right)$.

\item Query Size: This is inherited from the query size of the
  inner. Since, we use the inner obtained viaTheorem \ref{iterateInner}, the
  inner makes ${\cal O}(1)$ queries.

\item Alphabet: Again, follows from Theorem \ref{iterateInner} that alphabet
  is $\{0,1\}$.

  \item Randomness: The randomness of the composed verifier is the sum
    total of the randomness used by inner and that used by the
    outer. In our case, it follows from Theorem \ref{iterateInner} and Lemma
    \ref{lowlc} that the inner uses $\log n \cdot \alpha'$ random bits and the outer uses $\log n \cdot \left(1
      +o(1)\right)$. So, the total number of random bits is $\alpha' \cdot \log n + \left(1 +
      o\left(1\right)\right) \cdot \log n \ \ \simeq \ \ \left(1 +
      \alpha + o(1)\right) \cdot \log n$.
\end{itemize}
\qed



\subsubsection*{Soundness Amplification}

The tester obtained by Theorem \ref{constantSoundness} has a constant
soundness, for some arbitrary constant. We now amplify the soundness
of the tester whilst losing modest factors in the completeness.
Again, we would like to amplify the soundness whilst retaining the
good properties which are helpful in making the composition modular
even possibly at the expense of a non-binary alphabet. In fact, to
achieve non-constant soundness we would need larger alphabet. To this
end, we shall use Locally Decode and Reject Codes (Definition \ref{def:LDRC} of Moshkovitz and
Raz \cite{MR08}. To amplify the soundness, we use Theorem \ref{query}
on the tester obtained from Theorem \ref{constantSoundness}. The
construction follows.


\begin{theorem}[Linear PCP with Low Error]\label{main} 
For every $\alpha , \epsilon > 0$
\[
  3SAT \in PCP_{{1 - {n^{-\frac{\alpha'} {1 + \alpha'}}}}, \epsilon } \big[ (1 + o(1)) \cdot \log n,  2 \big ]_{\{0,1\}^{{\sf poly}(\frac{1}{\epsilon})}}
\]

\end{theorem}
\noindent {\em Proof.} Follows by invoking Theorem \ref{query} on the
tester obtained in Theorem \ref{constantSoundness}. We 
use the $( \q, \epsilon)$-construction algorithm for query
reduction and soundness amplification, where $\q$ is constant obtained
from Theorem \ref{constantSoundness}. We now analyze the parameters
of the PCP.
\begin{itemize}

\item {\em Completeness:} The completeness error of the PCP
  obtained in Thereom \ref{constantSoundness} is $n^\alpha$. Theorem \ref{query}
  introduces an error which is proportional to the soundness error of
  the PCP obtained by applying it. In fact, Theorem \ref{query} is
  essentially a parallel-repetition theorem in the low-error regime,
  albeit with a much worse alphabet tradeoff. Thus, the completeness
  error is that product of $n^\alpha \cdot \epsilon$. In other words,
  the error is not more than $n^{\alpha} \cdot \epsilon$, choose $ n^\alpha =
  \omega(\epsilon')$ to get the parameters we want.

\item {\em Soundness:} It follows from Theorem \ref{query} that the
  soundness of the new construction is $\epsilon$.

\item {\em Alphabet:} We inherit the alphabet of Theorem \ref{query},
  which happens to be $\{0,1\}^{{\sf poly}(\frac{1}{\epsilon})}$.
For poly alphabet, choose ${{\sf poly}(\frac{1}{\epsilon})} = \log n$ to get the 
necessary parameters.

\item {\em Randomness:} Theorem \ref{query} blows up the randomness of
  the input tester to ${1 + o(1)} \cdot {\sf R}$, where ${\sf R}$ is
  randomness of the PCP obtained by Theorem \ref{constantSoundness}. For the
  parameters that we have chosen, it is $(1 + o(1))  \cdot (1 + \alpha + o(1)) \cdot \log n   
\simeq  \log n \cdot (1 +  \alpha + o(1))$.
\end{itemize}
\qed


\begin{corollary}[{\sc Label-Cover}] \label{newcover}
  For every $n$, $\alpha$ and some $\beta > 0$ the following
  holds. Solving an instance of ${\sc SAT}$ of size $n$ can be reduced
  to distinguishing {\sc Label-Cover} instances 
  between the following two cases.
\begin{itemize}
\item {\sf Yes:} There is a labeling that satisfies at least $(1 -
		\frac{1}{n^{\alpha'}})$-fraction of the edges.
\item {\sf No:} Any labeling satisfies at most $(\frac{1}{(\log n)^\beta})$-fraction of the edges.
\end{itemize}
The size of the {\sc Label-Cover} instance is $n^{1 + \alpha}$ and every projection is linear in nature.
\end{corollary}

\begin{corollary}[{\sc 3Lin Hardness}] \label{3linpoly} For every
  $\alpha$, there is some $\beta > 0$, the following holds. Given a
  {\sc 3Lin} instance $\Phi$ of size $n$, it is {\sf NP}-hard to
  distinguish between the following two cases.
\begin{itemize}
\item {\sf Yes:} There is an assignment which satisfies $(1 -
		\frac{1}{n^{\alpha'}})$-fraction of $\Phi$.
\item {\sf No:} Every assignment can satisfy at most $(\frac{1}{2} +
    \frac{1}{(\log n)^{\beta}})$-fraction of $\Phi$.
\end{itemize}
\end{corollary}
\noindent {\em Proof.} Plugging in the {\sc Label-Cover} obtained from
Corollary \ref{newcover} instead of the one obtained by Corollary
\ref{oldcover} in Theorem \ref{3lin} by setting 
$\epsilon = \frac{1}{n^\alpha}$, and $\delta = \frac{1}{(\log
n)^{\sqrt[3]{\beta}}}$.  Completeness follows trivially from Theorem
\ref{khot}. Since $S < \delta^2$, the soundness of $V_{lin}$ is $\frac{1}{2} +
\frac{1}{(\log n)^{\beta'}}$. This completes the proof.\qed

\begin{corollary}[{\sc Min-3Lin-Deletion Hardness}] \label{3delpoly}
  For every $\alpha$ and some $\beta > 0$ the following holds. Given a
  {\sc Max-3Lin-Deletion} ${\cal D}$ instance of size $n$ it is {\sf
    NP}-hard to distinguish between the following two cases.
\begin{itemize}
\item {\sf Yes:} There is a $(\frac{1}{n^{\alpha'}})$-fraction of
  equations whose removal makes ${\cal D}$ satisfiable.
\item {\sf No:} At least $\beta$-fraction of the equations in ${\cal
    D}$ need to be removed from it to make it satisfiable.
\end{itemize}
\end{corollary}
\noindent {\em Proof.} An alternate view of Corollary \ref{3linpoly}. \qed \\

\begin{corollary}[{\sc 3Sat Hardness}]
  For every $\alpha > 0$ and some $\beta > 0$, given a {\sc 3Sat}
  instance $\Upsilon$ of size $n$ it is {\sf NP}-hard to distinguish
  between the following two cases.
\begin{itemize}
\item {\sf Yes:} There is an assignment which satisfies $(1 -
    \frac{1}{n^{\alpha'}})$-fraction of $\Upsilon$.
\item {\sf No:} Every assignment can satisfy at most $(\frac{7}{8} +
    \frac{1}{(\log n)^{\beta}})$-fraction of $\Upsilon$.
\end{itemize}
\end{corollary}





















