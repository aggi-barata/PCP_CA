\chapter{Tensoring}
\section{Preliminaries}

Throughout this paper we work with the binary alphabet $\Sigma =
\{0,1\}$, and arithmetic over the alphabet is over the field $GF(2)$.

\subsection{Hamming Weight}

We start by introducing the notion of Hamming weight and its
corresponding metric the Hamming distance. For any ${\bf x} \in
\Sigma^n$, the hamming weight of {\bf x} is the number of nonzero
coordinates of {\bf x}, also known by $wt({\bf x})$. The weight
function is a norm and the corresponding metric $d({\bf x, y})$ is the
weight of difference between {\bf x} and {\bf y}, that is, $wt({\bf x}
+ {\bf y})$ . It is often useful to normalize the Hamming weight
(resp. distance) w.r.t. $n$. From now on, we use normalized Hamming
weight, which is wt(${\bf x}$)$/n$. Also, we use bold face small case
letters {\bf x}, {\bf y} to denote column vectors and upper case bold
letters like {\bf A}, {\bf X} to denote matrices.  We use the notation
({\bf $\cdot$ })$^T$ to denote the transpose of the corresponding
vector or matrix.

\begin{claim} \label{ind} For every {\bf x, y, z}, d({\bf x}, {\bf y})
  = $d({\bf x} + {\bf y}, {\bf y} + {\bf z})$.
\end{claim}
\noindent {\em Proof.} Say that {\bf x, y} agree on a coordinate $i$,
then they continue to remain so even after adding ${\bf z}$ to both
{\bf x, y}. Similarly, if {\bf x,y} disagree on the coordinate. \qed

\subsection{Linear Systems of Equations}

Let $\eL,\ \eL \ \equiv {\bf Ax} + {\bf y} = {\bf 0}$, be a linear
system with $m$ equations over $n$ variables ${\bf x}$. We use the notation
UNSAT($\eL$, {\bf x}) to denote the fraction of unsatisfied equations
in $\eL$ when its variables are assigned to {\bf x}. Note that, this
is exactly same as the Hamming weight of ${\bf Ax} + {\bf y}$. We
define the UNSAT value of $\eL$ as follows.


\begin{definition}[UNSAT($\eL$)] The UNSAT value of $\eL$ is defined
  to be minimum fraction of unsatisfied equations over all possible
  assignments to the variables.
\end{definition}


\begin{proposition} \label{unsat}
The following holds:
\begin{equation}
                \mbox{UNSAT(}\eL\mbox{)} = \min_{{\bf x}} d({\bf Ax, y}) = \min_{{\bf x}} d({\bf Ax + y, 0}) \label{UNSATEQ}
\end{equation}
\end{proposition}
\noindent {\em Proof.} Recall that, UNSAT value of $\eL$ is the
minimum Hamming weight of {\bf Ax +y} over all ${\bf x}$. This settles
that the first and the last expression are the same. Now, subtract
{\bf y} from both terms of $d$({\bf Ax + y, 0}). We now invoke Claim
\ref{ind} to establish that shifts preserve Hamming distances.
And this establishes the truth of \eqref{UNSATEQ}. \qed \\


\section{Tensoring} \label{section:Tensoring}

We start by defining the tensor of $n$-dimensional vector
${\bf x}$ over a binary alphabet.

\begin{definition}[Tensor]\label{Tensoring}
  We define the tensor of a vector ${\bf x} \in \{0,1\}^n$ as the $n\times n$ matrix ${\bf
    x}^{\otimes 2}$ that has in its $i,j$ position the entry $x_i\cdot x_j$.
\end{definition}
We often treat matrices as vectors, so an $n\times n$ matrix corresponds to a vector of length $n^2$.

The tensor $\eL^{\otimes 2}$ of the system $\eL$ involves
taking the products of all possible pairs of (left hand sides of) linear equations from $\eL$,
and then linearizing them via replacing terms of the form $x_i \cdot
x_j$ with $x_{ij}$ and $x_i$ with $x_{ii}$ (recall that in $GF(2)$ it holds that $x_i^2 = x_i$). The claim that we make about the aforementioned
transformation is that the UNSAT values of $\eL$ and its tensor are
very closely related. In what follows, we make this connection
explicit. 

\begin{definition}
  Given a linear system $\eL$, $\eL \equiv {\bf Ax + y = 0}$, over variables 
  ${\bf x} = (x_1, \ldots x_n)$, the tensor of
  $\eL$ is defined in two steps: (1) Consider the quadratic system $\left({\bf Ax + y}\right) \cdot \left({\bf Ax +
      y} \right)^T$. (2) Linearize the quadratic system by replacing every term of the form $x_i \cdot x_j$ for
  $i, j \in [n]$ by $X_{ij}$, and every term of the form $x_i$ for $i\in [n]$ with $X_{ii}$. We denote this linearized system by $\eL^{\otimes
    2}$. We use the matrix ${\bf X}$ to denote the variables $X_{ij}$ of this system.
\end{definition} 
As a matter of convention, we associate with ${\bf X}$, the variables of the tensored system, 
its diagonal ${\bf x} = (X_{11},\ldots,X_{nn})$, as a candidate assignment to the original system. 
Note that when ${\bf X} = {\bf x}^{\otimes 2}$, it indeed holds that ${\bf X}$'s diagonal is ${\bf x}$.
Using this convention, we can develop an expression for the system $\eL^{\otimes 2}$ as a 
function of the ingredients of $\eL$, i.e., the matrix ${\bf A}$ and the vector ${\bf y}$:
\begin{eqnarray}\label{eq:tensored-system}
{\bf AXA}^T + {\bf y}({\bf Ax})^T + \left({\bf Ax} + {\bf y} \right)\cdot {\bf y}^T
\end{eqnarray}


\begin{proposition}\label{basic}
  For every linear system $\eL$,
\[
    \mbox{UNSAT}(\eL^{\otimes 2}, {\bf x}^{\otimes 2}) = \mbox{UNSAT}(\eL, {\bf x})^2.
\]
\end{proposition}
\noindent {\em Proof.} When ${\bf X} = {\bf x}^{\otimes 2}$, the tensored system $\eL^{\otimes 2}$ is the same as the qudratic system from which it was generated. An equation in the quadratic system is not satisfied if and only if both of the equations it consists of were not satisfied. The relation stated in the proposition follows. \qed \\

%\begin{claim} \label{claim:linearTensorProperty}
%For every {\bf A}, {\bf X}, the rows and columns of the linear 
%form ${\bf AXA}^T$ are equal to {\bf Ax}, for some {\bf x}.
%\end{claim}
%\noindent {\em Proof.}  Follows from the fact that the columns and rows of a %tensored code	
%are the codewords of code being tensored \cite{ECC}. \qed

\begin{lemma}\label{converse} For every linear system $\eL$ over variables ${\bf X}$ with diagonal ${\bf x}$,
$$\mbox{UNSAT}(\eL^{\otimes 2},{\bf X}) \geq \mbox{UNSAT}(\eL, {\bf x})\cdot \mbox{UNSAT}(\eL).$$
In particular, $\mbox{UNSAT}(\eL^{\otimes 2}) \geq \mbox{UNSAT}(\eL)^2$.
\end{lemma}
\noindent {\em Proof.} 
Let us use the following notation for the two parts of $\eL^{\otimes
  2}$ as in Equation~(\ref{eq:tensored-system}):
$${\bf P}:= {\bf AXA}^T + {\bf y}({\bf Ax})^T;\;\;\; {\bf Q}:= ({\bf Ax} + {\bf y}){\bf y}^T.$$
We start by analyzing {\bf Q}. For every $i\in [n]$, the $i$'th row of
${\bf Q}$ is either {\bf 0}, if $({\bf Ax} + {\bf y})_i = 0$, or {\bf
  y}, if $({\bf Ax} + {\bf y})_i =1$.

Consider {\bf P}. We observe that the rows of ${\bf AXA}^T$ are all in
the image of ${\bf A}$, and so are the rows of ${\bf y}({\bf
  Ax})^T$. It follows from linearity that all of ${\bf P}$'s rows are
in the image of ${\bf A}$.

Thus, for every $i\in [n]$ such that $({\bf Ax} + {\bf y})_i =1$, the
$i$'th row of ${\bf P + Q}$ is of the form ${\bf A x_i} + {\bf y}$ for
some ${\bf x_i}$.  Overall, for at least $\mbox{UNSAT}(\eL,x)$ of the
rows, there are at least $\mbox{UNSAT}(\eL)$ non-zeros. \qed 

\eat{
\subsection{Decoding Tensored Systems}

So far, we were essentially recollecting well known facts. In this
work, we would like to use tensoring to construct a PCP. In any PCP
theorem, the challenging part is to infer some structure if we accept
certain proof. Informally, this corresponds to establishing that if we
accept a purported proof, we must be able to decode an assignment that
(approximately) achieves the claims made by the purported proof. For
our use of tensoring, we would like to state that if an assignment
${\bf X}$ maximally satisfies the tensored system, then necessarily
${\bf X} = {\bf x}^{\otimes 2}$ when ${\bf x}$ is the diagonal of
${\bf X}$. We only focus on symmetric matrices ${\bf X}$. We can
enforce this using a technique known as {\em folding} in the proof
verification literature.

\subsubsection{Folding} 

We can restrict our ${\bf X}$'s to symmetric matrices, that is,
${X}_{ij} = {X}_{ji}$ for all $i,j\in [n]$. This is done simply by not
having separate entries for $X_{ij}$ and $X_{ji}$, but only one entry,
say $X_{ij}$. The term $X_{ji}$ is replaced by $X_{ij}$.

\subsubsection{Observations}

The following observations will be crucial for our analysis:

\begin{proposition}\label{symmetric}
For every symmetric matrix {\bf M}, if the Hamming weight of {\bf M} is $k^2$ and
there is a subset of $k$ rows each with a Hamming weight of $k$, then it must be that
{\bf M} has only $k$ non-zero columns. Moreover, the Hamming weight
of each of these columns is $k$.
\end{proposition}
\noindent {\em Proof.} Assume for the sake of contradiction that {\bf M}
has $k+1$ non-zero columns. From the symmetry of {\bf M}, it follows that
{\bf M} has $k + 1$ non-zero rows. Since, {\bf M} already has $k$ rows each with Hamming weight 
$k$, an additional non-zero row would imply that the Hamming weight of {\bf M} is larger than $k^2$ 
and this contradicts the hypothesis. Thus, it must be that {\bf M} has only $k$ non-zero 
columns. The artifact that each of these columns has a Hamming weight of $k$ follows from symmetry. \qed


\begin{proposition}[Uniqueness]\label{unique}
For every {\bf x} $\in {GF}_2^n$, every row of {\bf x}$^{\otimes 2}$ is 
either equal to {\bf x} or {\bf 0}. Moreover, {\bf x}$^{\otimes 2}$ is symmetric.
\end{proposition}
\noindent {\em Sketch.} Follows from the definition of {\bf
  x}$^{\otimes 2}$ and that we are working with ${GF}_2$. \qed

\begin{proposition}[Converse of Uniqueness]\label{converseUnique}
  For every symmetric {\bf X}, if every row of {\bf X} is either {\bf
    x} or {\bf 0}, then {\bf X} = {\bf x}$^{\otimes 2}$.
\end{proposition}
\noindent {\em Proof.} Say {\bf X} has {\bf x} over rows $R$. By
symmetry, {\bf X} also has them over $R$ columns. Thus, {\bf X} has
$1$ in the ($i,j$) entry whenever {\bf x} has $1$ at both $i$-th and
$j$-th entries and zero elsewhere. And this is exactly the definition
of {\bf x}$^{\otimes 2}$.\qed\\

\noindent For any linear system $\eL \equiv {\bf Ax} + {\bf y} = {\bf 0}$, we define d($\A$) as
the $\min_{{\bf x} \ne {\bf 0}} {\bf Ax}$.\footnote{This is exactly
  the distance of the linear code having {\bf A} as its generator
  matrix.}

\begin{claim}
  For every {\bf A} and {\bf x}$^{\otimes 2}$, the linear form ${\bf
    Ax}^{\otimes 2}{\bf A}^T$ has every row equal to either {\bf Ax}
  or {\bf 0}.
\end{claim}
\noindent {\em Sketch.} One may rewrite ${\bf Ax}^{\otimes 2}{\bf A}^T$  as 
${\bf Ax}\cdot ({\bf Ax})^T$ . Thus, the product is of the form $({\bf  Ax})^{\otimes 2}$ 
and now the claim follows from Proposition \ref{unique}. \qed



\begin{claim} \label{NonUnique}
For every {\bf A} and symmetric {\bf X}, if every row of the linear 
form ${\bf AXA}^T$ is either {\bf Ax} or {\bf 0} and d($\A$) $ > 0$, then 
{\bf X} = {\bf x}$^{\otimes 2}$.
\end{claim}
\noindent {\em Proof.} Since d($\A$) $> 0$, for any ${\bf v_1}$ different from ${\bf v_2},\ {\bf Av_1} \ne {\bf Av_2}$. 
Given this observation, for any ${\bf z}$ in the image of {\bf A}, one
can uniquely determine ${\bf x_0}$ such that ${\bf Ax_0} = {\bf z}$. 
Now, the Claim follows via similar arguments as in Proposition 
\ref{converseUnique}. \qed\\

	
\begin{claim} \label{claim:Punique}
For every linear system $\eL, \eL \equiv {\bf Ax} + {\bf y} = {\bf 0}$, and any symmetric {\bf X},
if every row of ${\bf P}, {\bf P} = {\bf AXA}^T + {\bf y}  
\cdot ({\bf Ax})^T$, is either {\bf Ax} or {\bf 0} and d($\A$)  $> 0$, 
then {\bf X} $=$ {\bf x}$^{\otimes 2}$. 
\end{claim}
\noindent {\em Proof.}  We start by observing that if every row of
{\bf P} is either {\bf Ax} or {\bf 0}, then the sum of ${\bf P}$ and
${\bf y}({\bf Ax})^T$ also has each of its row either equal to {\bf
  Ax} or {\bf 0}.  The last observation follows from the fact that we
are working with ${GF}_2$ and it is easy to see that the set
$\{{\bf Ax}, {\bf 0}\}$ is closed under addition. Thus, {\bf Z}, {\bf
  Z} = ${\bf P} + {\bf y}({\bf Ax})^T$, has every row either equal to
{\bf Ax} or {\bf 0}.  Recall that {\bf Z} is essentially the linear
form ${\bf AXA}^T$, for some {\bf X}. Since d($\A$) $> 0$,  we may invoke
 Claim \ref{NonUnique} to infer that {\bf X} must be equal to {\bf x}$^{\otimes 2}$.  \qed

\begin{proposition}\label{Non-equality}
For any non-zero {\bf v} and every {\bf X}, linear 
system $\eL, \ \eL$ = {\bf Ax} $+$ {\bf y} = {\bf 0}, such 
that UNSAT($\eL$) $> 0$ and d($\A$) $ > 0$, 
it holds that  ${\bf AXA}^T  \ne {\bf y} \cdot ({\bf Av})^T$.
\end{proposition}
\noindent {\em Proof.}  It follows from linearity that the 
columns of ${\bf AXA}^T$ are in the image of
{\bf A}. Also from hypothesis, since {\bf v} is non-zero and d($\A$) $> 0$, 
it follows that ${\bf Av}$ is 
a non-zero vector. Thus, at least one column of ${\bf y} \cdot ({\bf Av})^T$ is 
equal to {\bf y}. We now use Proposition \ref{unsat} over that column to conclude that for any {\bf z}, {\bf Az} and {\bf y} differ
on at least one of the coordinates. Thus, for any {\bf v} different from {\bf 0} and every {\bf X} and $\eL$,  
it holds that ${\bf AXA}^T  \ne {\bf y} \cdot ({\bf Av})^T$, unless d($\A$) = 0.  \qed


\subsubsection{Main Lemma}

\noindent We now show that if the prover provides us with any
assignment ${\bf X}$ other than one equal to {\bf x}$^{\otimes 2}$, then
UNSAT($\eL^{\otimes 2}$, {\bf X}) is larger than $d^2$. 

\begin{lemma}[Unique Tensor Lemma]\label{decode}
  For any linear system $\eL, \eL \equiv {\bf AX} + {\bf y} = {\bf
    0}$, with UNSAT value $d$ and every symmetric ${\bf X}$, if
  UNSAT($\eL^{\otimes 2}, {\bf X}$) = $d^2$, then {\bf X} = {\bf
    x}$^{\otimes 2}$ unless d($\A$) = 0.% Moreover, the
%  UNSAT($\eL$, {\bf x}) = UNSAT($\eL, \tilde{\bf x}$) = $d$.
\end{lemma}
\noindent {\em Proof.} We begin by recalling that for every {\bf X}
\[
\mbox{UNSAT}(\eL^{\otimes 2}, {\bf X}) = \mbox{wt}({\bf AXA}^T + {\bf
  y}({\bf Ax})^T + \left({\bf Ax} + {\bf y} \right)\cdot {\bf y}^T)
\]

\noindent As a matter of convention, we denote the first two terms of the RHS 
by {\bf P} and the latter by {\bf Q}. Let us 
denote the indices of the non-zero rows of {\bf Q} by $\s$. Recall Lemma 
\ref{converse} to infer that wt$({\bf Ax} + {\bf y})$ must be $d$. 
Thus, $|\s| = d$. Also, the Hamming weight of {\bf P} + {\bf Q} over the
rows indexed by $\s$ is at least $d^2$. Let ${\overline{\s}}$ denote 
the set of indices of all non-zero rows of {\bf P} such that the 
corresponding row of {\bf Q} is equal to {\bf 0}. \\


\noindent For the Hamming weight of {\bf P} + {\bf Q} to be 
equal to $d^2$, it must be that $\overline{\s} = \emptyset$. For 
otherwise, since the sets $\s$ and $\overline{\s}$  are disjoint, the 
Hamming weight of {\bf P} + {\bf Q} is at least the sum total of Hamming
weight over rows with indices $\s$ and $\overline{\s}$, which is greater 
than $d^2$ unless $\overline{\s} = \emptyset$.\footnote{Because, the non-zero rows of 
{\bf Q} alone contribute to a weight of $d^2$ in {\bf P} + {\bf Q} while $\overline{\s}$ deals with 
only the rows of {\bf Q} which are equal to {\bf 0}.} Thus, if the 
Hamming weight of {\bf  P} + {\bf Q} is $d^2$, then the 
weight of {\bf P} cannot be outside the non-zero rows of {\bf Q}.\\

\noindent We now show that if each of the rows in {\bf P} indexed by $\s$ ain't the 
same, then the Hamming weight of {\bf P} + {\bf Q} is strictly larger than $d^2$. In what follows
we shall deal with abolute Hamming weight rather than its normalized definition. Say, there are two rows ${\bf A}\hat{\bf
    x}$ and {\bf A}$\tilde{\bf x}$ different from one another. Since 
${\bf A}\hat{\bf  x}$ and {\bf A}$\tilde{\bf x}$ are different from one another, they must differ on at least one coordinate.
Also, since the indices of these rows are in $\s$, they are both added to {\bf y}.
We now observe via Claim \ref{ind} that ${\bf A}\hat{\bf
    x} + {\bf y}$ and {\bf A}$\tilde{\bf x} + {\bf y}$ must also
  differ on at least one coordinate. Moreover, by Proposition
  \ref{unsat} any {\bf Ax$_0$} + {\bf y} must have at least $dm$ non-zero
  coordinates, where $m$ is the dimension of {\bf y}. Hence, there are 
at least $dm + 1$ non-zero coordinates over
  both ${\bf A}\hat{\bf x} + {\bf y}$ and {\bf A}$\tilde{\bf x} + {\bf
    y}$. Therefore, {\bf P} + {\bf Q} must have at least
  ($dm + 1$) non-zero columns. Since {\bf P} + {\bf Q} is symmetric, by 
the contrapositive of Proposition \ref{symmetric}, if {\bf P} +{\bf Q}
has ($dm + 1$) non-zero columns, then either the Hamming weight of {\bf P} + {\bf Q} 
is not $d^2 \cdot m^2$ or {\bf P} + {\bf Q} does not have $d \cdot m$ rows each with 
Hamming weight $d \cdot m$. Since, the rows of {\bf P} + {\bf Q} indexed by $\s$ are in fact 
$d \cdot m$ rows each with a Hamming weight of $d \cdot m$, it must be that 
the Hamming weight of {\bf P} + {\bf Q} is not $d^2 \cdot m^2$. Hence, if {\bf P} does 
not have the same row over the rows indexed by $\s$, the Hamming weight of 
  {\bf P} + {\bf Q} is strictly greater than $d^2 \cdot m^2$. In other words, if the
(normalized) Hamming weight of {\bf P} + {\bf Q} is $d^2$, then {\bf P} has the 
same row across every row indexed by $\s$.  \\


\noindent In short, we have established that {\bf P} must have the same row across
all indices in $\s$ and every row outside $\s$ is {\bf 0}. {\em w.l.o.g.}, assume that one of the following is true: 
(1) {\bf P} has {\bf Ax} as its row across all rows indexed by $\s$. (2)
{\bf P} has a row different from {\bf Ax}.

\begin{enumerate}
\item [Case 1:] Since every row of {\bf P} is either {\bf Ax} or {\bf 0}
and d($\A$) $> 0$, by Claim \ref{claim:Punique}, {\bf X} = {\bf x}$^{\otimes
  2}$. 
 
\item [Case 2:] Assume for the sake of contradiction that {\bf P} has a row ${\bf
    A}\tilde{\bf x}$ different from {\bf Ax}. Also  since 
${\bf A}\tilde{\bf x}$ different from {\bf Ax}, we may deduce that 
${\bf x}$ is different from  $\tilde{\bf x}$.  Now, {\bf P} + {\bf Q} can be written
as.
\begin{align*}
  {\bf  P} + {\bf Q} &= ({\bf Ax} +
  {\bf y}) \cdot ({\bf A}\tilde{\bf x})^T + \left({\bf Ax} + {\bf y}
  \right)\cdot {\bf y}^T \\
  & = {\bf A} \cdot {\bf x} \cdot \tilde{\bf x}^T \cdot
  {\bf A}^T + {\bf y}\cdot \left({\bf A}\tilde{\bf x}\right)^T +  \left( {\bf Ax} + {\bf y} \right)\cdot {\bf y}^T
\end{align*}

But, the above expression must be exactly equal to 
 \[
 {\bf AXA}^T + {\bf y}({\bf Ax})^T + \left({\bf Ax} + {\bf y} \right)\cdot {\bf y}^T
\]

Therefore,
\begin{align*}
  {\bf AXA}^T + {\bf y}({\bf Ax})^T &= {\bf A} \cdot {\bf x} \cdot
  \tilde{\bf x}^T \cdot
  {\bf A}^T + {\bf y}\cdot \left({\bf A}\tilde{\bf x}\right)^T\\
  {\bf A} \cdot ({\bf X} + {\bf x} \cdot \tilde{\bf x}^T) \cdot {\bf A}^T &= {\bf y} \cdot \left({\bf A} \cdot \left({\bf x} + \tilde{\bf x}\right)\right)^T
\end{align*}
Since, {\bf x} is different from $\tilde{\bf x}$, it follows that
${\bf x} + \tilde{\bf x}$ is a non-zero vector. Therefore, we can invoke Proposition \ref{Non-equality}
to conclude that there is no $({\bf X} + {\bf x} \cdot \tilde{\bf x}^T)$ for which
the above holds. Hence, a contradiction. Therefore, {\bf P} cannot have 
only ${\bf A}\tilde{\bf x}$, for any ${\bf A}\tilde{\bf x}$ 
different from {\bf Ax}, in the rows indexed by $\s$.
\end{enumerate}

This concludes the proof of the Lemma as we have established that if 
UNSAT($\eL^{\otimes 2}, {\bf X}$) = $d^2$,
then ${\bf X} = {\bf x}^{\otimes 2}$ unless d($\A$) = 0.\qed 

\newpage}
% -------------------------    Excluded  ---------------------------------------------------------------------------------------------


\eat{




\eat{


Let us denote the set of rows whose indices are in $\s$ by
  {\bf P}$_\s$.




  Now consider the definition of {\bf P}, {\bf P} is obtained by
  adding {\bf AXA}$^T$ and ${\bf y} \cdot ({\bf Ax})^T$.  From the
  conclusions of the previous paragraph, we understand that {\bf
    P}$_\s$ is exactly equal to {\bf A}$\tilde{\bf x}$.  It is not
  hard to see that for ${\bf P}_\s$ to be {\bf A}$\tilde{\bf x}$, the
  $i$-th row of ${\bf AXA}^T$, denoted by $\{{\bf AXA}^T\}_i$, must be
  exactly the following.
\[
{\bf AXA}^T_i =
\left\{ \begin{array}{lcl}
 {\bf A}\cdot({\bf x} + \tilde{\bf x}) & \mbox{for}
& {\bf y}_i = 1,  \{{\bf Ax + y}\}_i  =1\\ {\bf Ax} & \mbox{for} &  {\bf y}_i = 1, \{{\bf Ax + y}\}_i = 0 \\
{\bf A}\tilde{\bf x} & \mbox{for} & {\bf y}_i = 0, \{{\bf Ax + y}\}_i =1 \\
{\bf 0} & \mbox{for} & {\bf y}_i = 0, \{{\bf Ax + y}\}_i = 0\\
\end{array}\right.
\]

At this point, note that {\bf A}, {\bf y}, {\bf x} and $\tilde{\bf x}$
are fixed. Given this setting, it easy to see that {\bf X} can be
computed exactly via Gaussian elimination. It now follows that {\bf X}
= $f({\bf A}, {\bf x}, \tilde{\bf x})$.  Thus, if the
UNSAT($\eL^{\otimes 2}$) = $d^2$, then {\bf X} is a function of {\bf
  A}, {\bf x} and $\tilde{\bf x}$.}


  \noindent A comment on the above lemma is due. Lemma \ref{decode}
  implies that if a tensored system satisfies a certain fraction of
  equations, then the tensored assignment is constructed from a unique
  assignment of $\eL$.  The key idea was that we set up ourselves such
  that we operate in the {\em unique decodability regime} of the
  tensored system. We summarize this via the following corollary.

\begin{corollary}
For every linear system $\eL$, $\eL \equiv {\bf Ax} + {\bf y} = {\bf 0}$, and {\bf X} with diagonal vector {\bf x}, unless d($\A$) = 0,
\[
 \mbox{UNSAT}(\eL^{\otimes 2}, {\bf X})  \le  \mbox{UNSAT}(\eL)^2    \implies {\bf X} = {\bf x}^{\otimes 2} 
\]
\end{corollary}


\paragraph{Linear Codes and Homogeneous Linear System.}

A homogeneous system of linear equations $\eL$ is one where setting
every variable to $0$ satisfies $\eL$. Also recall that a linear
system with $m$ equations over $n$ variables can be denoted by a
matrix ${\bf A \cdot x } = {\bf 0}, {\bf A} \in \mathbb{F}_2^{m\times
  n}$ and ${\bf x} \in {\mathbb F}^n_2q$.

The matrix ${\bf A}$ can also be conceived as the generator matrix of
a linear code \A = $\{ {\bf x \cdot A} | {\bf x} \in
\mathbb{F}^n_q\}$.  Thus, if $\eL$ has a non-trivial assignment,
meaning there is a non-zero ${\bf x}$ such that ${\bf Ax} = {\bf 0}$,
it corresponds to code with its minimum distance equal to
$0$. However, if $\eL$ has no non-trivial assignment satisfying all
equations, then its minimum distance is definitely non-zero. In
essence, there is a strong correlation between the (non-trivial)
satisfiability of $\eL$ and the minimum distance of the respective
linear code. We summarize our observations via the following
observation.

\begin{observation}
For every homogeneous system of linear equations $\eL \equiv {\bf Ax = 0}$,
the minimum distance of $\A$ is equal to the minimum fraction of 
unsatisfied equations over all possible (non-trivial) assignments
to variables.
\end{observation}

We extend the definition of tensoring of ${\bf x}$ to tensor of a
linear system $\eL$ in a natural way. We define the tensor of $\eL =
\{y_1 \ldots, y_m\}$ over $n$ Boolean valued variables as
$\eL^{\otimes 2} : [m] \times [m] \rightarrow y_i \cdot
y_j$. Additionally, we replace a bivariate variable $x_i \cdot x_j$
with $x_{ij}$ and $x_i$ with $x_{ii}$. Hence, $\eL^{\otimes 2}$ has
$m^2$ equations over $n^2$ Boolean variables.  


To understand if the correspondence mentioned in the prequel also
extends to the tensored system $\eL^{\otimes 2}$, we introduce direct
product of the code \cite{ECC}. %from Dumer, Micciancio and Sudan \cite{DMS03}.
\begin{definition}
  For $i \in \{1, 2\}$, let $\A_i$ be a linear code generated by ${\bf
    A_i} \in \mathbb{F}_2^{n_i \times m_i}$.  Then the direct product
  of $\A_1$ and $\A_2$, denoted by $\A_1 \otimes \A_2$ is a code over
  $\mathbb{F}_2$ of block length $m_1m_2$ and dimension
  $n_1n_2$. Identifying the set $\mathbb{F}_2^{m_2 \times m_1}$ of all
  $m_2 \times m_1$ matrices in the obvious way, the direct product
  $\A_1 \otimes \A_2$ is defined as the set of all matrices $\{{\bf
    A}_2^T {\bf XA}_1 | {\bf X} \in \mathbb{F}_2^{n_2 \times n_1} \}$
  in $\mathbb{F}_2^{m_2 \times m_1}$.
\end{definition}

Notice that the codewords of $\A_1 \otimes \A_2$ are matrices whose
rows are codewords of $\A_1$ and columns are codewords of $\A_2$. In
our note, we will need the following fundamental property of direct
product codes.

\begin{claim} \label{distance} For linear codes $\A_1$ and $\A_2$ of
  minimum distance $d_1$ and $d_2$, their direct product is a linear
  code of distance $d_1d_2$.
\end{claim}
\noindent {\em Proof.} The linearity of $\A^{\otimes 2}$ follows from
its definition.  First lets establish that any encoding of any
non-zero ${\bf X}$ has at least $d^2$ non-zero entries.  Consider
${\bf XA}$, whose rows are codewords of $\A$. Notice that each row (by
virtue of being a codeword of $\A$) has at least $d_1$ non-zero
entries. Thus, ${\bf XA}$ has at least $d$ non-zero columns.  Now,
turn the attention towards the product ${\bf A^{T}XA}$. At least $d$
columns of this matrix are non-zero and each of them has at least
$d_2$ non-zero entries. Thus, any non-zero ${\bf X}$ has a weight of
$d_1\cdot d_2$ or more.

Now, to verify that the minimum distance is exactly $d_1 \cdot
d_2$. Consider the set of vectors ${\bf x_i}$ such that ${\bf x_iA_i}$
has exactly $d_1$ non-zero elements. Notice that the matrix $M = {\bf
  A^{T}x_{2}^{T}x_1A_1}$ is a codeword of $\A^{\otimes
  2}$. Rearranging the terms to ${\bf (x_{2}A)^{T}(x_1A)}$, we observe
that $i$-th column is zero if the $i$-th coordinate of ${\bf x_1A_1}$
is zero and analogously for $j$-th coordinate of ${\bf x_2A}$. Thus,
$M$ is zero on all but $d_1 \cdot d_2$ entries. \qed



\begin{corollary}
  For any linear code $\A$ of minimum distance $d$, the code
  $\A^{\otimes 2}$ is a linear code of distance $d^2$.
\end{corollary}
\noindent {\em Proof.} Follows from Claim \ref{distance}. \qed

Observe that the tensored system $\eL^{\otimes 2}$ is exactly same as
$({\bf A} \otimes {\bf A}) {\bf X} = {\bf 0}$, where ${\bf AX = 0}$ is
$\eL^{\otimes 2}$.  In what follows, we further explore the properties
of a tensored system.

\paragraph{Affine Codes and Non-Homogeneous Systems.}

In the above section, we restricted our focus to linear systems which
have a trivial zero. We now shift gears to study a general class of
linear systems which has no trivial zeros. The nice relationship that
was pointed out in the last section is disturbed ever so slightly as
the codewords no longer form a linear subspace. However, they are
still closed under addition and scalar multiplication and thereby the
codewords form an {\em affine subspace}. Thus, the results and
observations made earlier do not follow as smoothly as they did.

\noindent We begin by defining an {\em affine shift} of a linear
code. For any vector ${\bf x} \in \mathbb{F}_2^{n}, \A \oplus {\bf x}$
is defined as
\[
        \A \oplus {\bf x} = \{{\bf x + y ~|~ y} \in \A \}.
\]

Thus, an affine code $\A$ is defined by a tuple $({\bf A, x})$. From
now on, we use the notation \A(${\bf A, x}$) to specify the code
generated as per the above definition. We now establish a useful claim
about the relation between an affine code and its corresponding linear
code generated its generator matrix.

\begin{claim} \label{ind}
The minimum distance of $\A \oplus {\bf x}$ is equal to the minimum distance of $\A$.
\end{claim}
\noindent {\em Proof.} The key observation is that the Hamming
distance of two vectors ${\bf y, z} \in {\mathbb{F}}_{2}^{n}$ remain
unchanged when a fixed vector ${\bf x}$ is added to {\bf y,
  z}. Specifically, say {\bf x, y} agree on a coordinate $i$, then
they continue to remain so even after adding $z \in \mathbb{F}_2$ to
both {\bf x, y}. \qed
% A one line proof for this claim in $\mathbb{F}_2$ is as follows:
% $d({\bf y, z}) = {\bf y \odot z}$, where $\odot$ is xor
% operation. Thus, $d({\bf y + x, z + x}) = {\bf y \odot x \odot z
%   \odot x}$, which is exactly ${\bf y \odot z}$. \qed


\subsection{Extending Direct Products to Affine Codes}
We, now, extend the definition of direct products of linear codes
defined earlier to affine codes in a natural way. An affine code is
determined by its generator matrix ${\bf A}$ and a displacement vector
${\bf y}$. We define a matrix ${\bf B^{(n)}}, n \ge 0$ with $n$ rows
and each of its rows being exactly equal to {\bf y}. The direct
product of an affine code $\A$ is defined as follows.

 \begin{definition}
   Let $\A$ be an affine code generated by (${\bf A, y})$ $\in
   \mathbb{F}_2^{n \times m}$.  Then the direct product of $\A$,
   denoted by $\A^{\otimes 2}$ is a code over $\mathbb{F}_2$ of block
   length $m^2$ and dimension $n^2$. Identifying the set
   $\mathbb{F}_2^{m \times m}$ of all $m \times m$ matrices in the
   obvious way, the direct product $\A$ is defined as the set of all
   matrices $\{ {\bf A}^T \left({\bf X A + B^{(n)} } \right) + {\bf
     B^{(m)^{T}}} | {\bf X} \in \mathbb{F}_2^{n \times n} \}$ in
   $\mathbb{F}_2^{m \times m}$.
 \end{definition}

 \begin{claim}
   The codewords of $\A ^{\otimes 2}$ are matrices whose rows and
   columns are codewords of $\A$.
 \end{claim} 
 \noindent {\em Proof.}  \remark{Yet to be fixed.} Let us start with
 the easier side -- the columns are codewords of $\A$. A quick glance
 at the outer multiplication shows that the matrix ${\bf A^{T}}$. One
 may treat ${\bf C}$ as a vector comprising of $[{\bf c_1 \ldots
   c_m}]$, where each of ${\bf c_i}$ is a column vector of length
 $n$. It is obvious that after the outer multiplication the columns of
 the resulting matrix are encodings of ${\bf c_i}$'s minus the parity
 shift. But, the addition with ${\bf B^{(m)^{T}}}$ serves the
 purpose. Thus, the columns are codewords of $\A$.

 For establishing that the rows are also the codewords of $\A$. By
 arguments akin to previous paragraph, the inner multiplication and
 addition with parity matrix results yields a matrix with each of the
 rows is a encoding of the corresponding row of ${\bf X}$. In fact,
 the resulting matrix is the matrix ${\bf C}$ mentioned in the
 prequel. \qed



\begin{claim}\label{fin}
For any affine code $\A$ of minimum distance $d$, the code $\A^{\otimes 2}$ is an affine code of
distance $d^2$
\end{claim}
\noindent{\em Proof.} The direct product operation can be rewritten aѕ
${\bf A^T X A} + {\bf A^T B^{(n)}} + {\bf B^{(m)}}$.  Notice that only
the first term varies and the rest is a constant that depends on the
chosen code.  Thus, the latter part is an displacement vector of
dimension $m^2$. By Claim \ref{ind}, the distance of the code is
determined entirely by the term ${\bf A^T X A}$. Notice that distance
of code generated by ${\bf A^T X A}$ is exactly $d^2$. For
completeness, we prove it in subsequent paragraphs.  At the outset,
notice that the code generated by ${\bf A^TXA}$ is linear. \qed


\subsection{Tensoring Non-Homogeneous Systems}

Let $\eL$ be a linear system with $m$ equations over $n$
variables. Say, $\eL^{\otimes 2}$ is its tensored system. We now show
the connection between $\eL^{\otimes 2}$ and the direct product of
affine codes. Elaborating further, if $\eL \equiv {\bf Ax + y = 0}$,
where ${\bf A}$ is the coefficient matrix of equations in $\eL$, ${\bf
  y}$ is vector of dimension $m$ denoting the constant term in every
equation.

\begin{claim} \label{same} For any linear system $\eL \equiv {\bf Ax +
    y} = {\bf 0}$ and its tensor $\eL^{\otimes 2}$, $i \in [m] \times
  [m]$, the coefficient of the variable $x_{j}, j \in [n] \times [n]$,
  is equal to the entry in the $i$-th row and $j$-th column of ${\bf
    A}^{\otimes 2}$ and the constant term in equation $i$ is equal to
  the $i$-th entry of {\bf y}$^{\otimes 2}$. Moreover, ${\bf
    A}^{\otimes 2}\left({\bf X} + {\bf y}^{\otimes 2}\right)$ is
  exactly ${\bf A^TXA} \oplus {\bf y}$.
\end{claim}
\noindent {\em Proof.} Essentially follows from the definitions of
tensoring and direct product. \qed


\begin{definition}[UNSAT($\eL$)] The UNSAT value of $\eL$ is
defined to be minimum fraction of unsatisfied equations over 
all possible assignments to the variables.
\end{definition}

%We state one useful fact about the UNSAT of $\eL \equiv {AX + y} = {\bf 0}$.

\begin{proposition} \label{unsat}
The following are equivalent.
\begin{equation}
                \mbox{UNSAT(}\eL\mbox{)} = \min_{{\bf x} \in \A} d({\bf x, y}) = \min_{{\bf x} \in \{0,1\}^n} d({\bf Ax + y, 0}) \label{UNSATEQ}
\end{equation}
\end{proposition}
\noindent {\em Proof.} Recall that, UNSAT value of $\eL$ is the
minimum Hamming weight of {\bf Ax +y} over all ${\bf x}$. This settles
that the first and the last expression are the same. Now, subtract
{\bf y} from both terms of $d$({\bf Ax + y, 0}). Lemma \ref{ind}
establishes that an affine shift preserves Hamming distances. This
inference establishes the truth of Equation \eqref{UNSATEQ}. \qed

\begin{proposition}[Tensoring vs Solvability]\label{basic}
For any linear system $\eL$ and its tensored system SAT($\eL^{\otimes 2}$),
\[
    \mbox{UNSAT}(\eL) = \rho \Rightarrow \mbox{UNSAT}(\eL^{\otimes 2})   \ge \rho^2  
%\item UNSAT($\eL^{\otimes 2}$) $ > \rho \Rightarrow$ SAT($\eL$) $\ge \sqrt{\rho}$.  
\]
\end{proposition}
\noindent {\em Proof.} We invoke Claims \ref{unsat} and \ref{fin}
to establish the lemma. Say we are given an assignment $\pi : [n]
\rightarrow \{0,1\}$ that satisfies ($1 - \rho$)-fraction of $\eL$, we
claim that the assignment $\tilde{\pi} : [n] \times [n] \rightarrow
\pi(i) \times \pi(j)$ satisfies at least ($1 - \rho^2$)-fraction of
$\eL^{\otimes 2}$. It follows from the definition of $\eL^{\otimes 2}$
that the only case an equation $\tilde{y}_{ij}$ from $\eL^{\otimes
  2}$ is unsatisfied iff $\pi$ does not satisfy both $y_i$, $y_j$ in
$\eL$. By a simple counting argument, we can check that there only
$\rho^2$-fraction of such pairs. And thus the lemma. \qed \\


\noindent Proposition \ref{basic} was essentially restating a folklore
theorem. In this work, we would like to use these operations to
construct a PCP. In any PCP theorem, the challenging part to infer
some structure if we accept certain proof. Informally, this
corresponds to establishing that if we accept a purported proof, we
must be able to decode an assignment that (approximately) achieves the
claims made by the purported proof. If we would like to use tensoring
to any end in PCP's, we would need a certain converse of Proposition
\ref{basic}. In other words, we would like to state something along
the lines that the satisfiability of the tensored system is well
behaved. We warm up via some nice, interesting observations.

\begin{claim}\label{uni}
  Let ${\bf x \in \{0,1\}^n}$ and denote ${\bf x}^{\otimes 2}$ by
  ${\bf X}$. Every row of ${\bf X}$ is either the vector ${\bf x}$ or
  {\bf 0}. Moreover, {\bf X} is symmetric.
\end{claim}
\noindent {\em Proof sketch.} Follows from the definition of $X$ and
the fact that we are working with $\mathbb{F}_2$.  \qed

 
\paragraph{Folding.} In what follows from now, we only focus on
symmetric matrices ${\bf X}$. In the world of $\eL^{\otimes 2}$ this
translates to restricting ourselves to assignments which are
symmetric, that is, ${\bf X}_{ij} = {\bf X}_{ji}$. This can be easily
enforced by reading an ${\bf X}_{ij}$ only after ordering of $[n]$ in
a canonical fashion. In the world of PCP's, this can viewed as {\em
  folding} the proof tables.


\begin{lemma}\label{unique}
  For every linear system $\eL$, $\eL \equiv \{y_1 \ldots y_m\}$, with
  UNSAT value greater than $d$ and any two distinct $n$ dimensional
  vectors ${\bf x},\ {\bf y}$ over $\mathbb{F}_2$, then the UNSAT
  value of $\eL\left({\bf x} \oplus {\bf y}\right)$ must be at least
  $(d+ 1)$.
\end{lemma}
\noindent {\em Proof sketch.} Since ${\bf x, y}$ are distinct, it is
immediate that they do not satisfy the same subset of equations.  For
otherwise, it follows from the linearity of constraints that the UNSAT
$\eL\left({\bf x} \oplus {\bf y}\right)$ is $0$.  In other words,
$\left({\bf x} \oplus {\bf y}\right)$ is a satisfying assignment to
$\eL$ and this contradicts the hypothesis. Thus, the symmetric
difference of the constraints satisfied by {\bf x, y} is
non-zero. Hence,
\[
      \mid i\ \vert \ y_i\left({\bf x}\right) \ \vee y_i\left({\bf y}\right) \mid \ > \ \mid \ i \ \vert \  y_i\left({\bf x}\right) \ \wedge y_i\left({\bf y}\right) \mid \\
\]
This helps us conclude that the UNSAT value of $\eL\left({\bf x}
  \oplus {\bf y}\right)$ must be at least ${d+ 1}$. \qed


\begin{lemma}\label{decode1}
  For every linear system $\eL$, $\eL \equiv {\bf Ax + y} = {\bf
    0}\}$, with UNSAT value greater than $d$, if there is an ${\bf X}$
  such that UNSAT$\left(\eL^{\otimes 2}\left({\bf X}\right)\right)$ is
  less than $d \cdot (d+1)$, then ${\bf X}$ can be written
  as ${\bf x}^{\otimes 2}$, for some unique ${\bf x} \in \{0,1\}^n$.
\end{lemma}
\noindent {\em Proof.} Before we proceed to prove this lemma. Observe
that is sufficient to prove that ${\bf X}$ is composed of unique {\bf
  x}. And this along with the fact that ${\bf X}$ is symmetric, allows
us to invoke Claim \ref{uni} to conclude that ${\bf x}^{\otimes 2} =
{\bf X}$. Hence, it suffices to show that every row in{\bf X} is
either the zero vector or {\bf x}.

We prove the contrapositive, that is, if ${\bf X}$ has two non-zero rows which
are different then UNSAT$\left(\eL^{\otimes 2}\left({\bf
      X}\right)\right)$ is at least $d \cdot \left(d +
  1/m\right)$. Let {\bf X} = \Big\{${\bf \left(x_1\right)^T,
  \left(x_2\right)^T \ldots \left(x_n\right)^n}$\Big\}, where for each
$i$, $x_i$ is an $n$ dimensional vector in $\{0,1\}$. Recall Claim
\ref{same} and observe that the operation $\eL^{\otimes 2}\left({\bf
    X} \right)$ can be rewritten as $\left({\bf A^T} \cdot \left({\bf
      X A} + {\bf y}\right) + {\bf y} \right)$. So, there are two
matrix multiplications {\bf X} is multiplied by $A$, followed by
addition with ${\bf y}$ and this is in-turn multiplied by ${\bf
  A^T}$,added to ${\bf y}$.

Lets begin analyzing step by step. In the first multiplication(${\bf X
  \cdot A}$), we are essentially evaluating $\eL$ over each vector
${\bf x_i}$ of {\bf X}. By hypothesis there are two distinct indices
$i,j$ such that ${\bf x_i \ne x_j}$ and $\eL$ has an UNSAT value is
greater than $d$.  We now invoke Lemma \ref{unique} to infer that
$\left({\bf A X} + {\bf y}\right)$, denoted by $\tilde{\bf X}$ has
least $(d+1)$-fraction of constraints that are not satisfied.
Alternatively, at least $\left(d +1\right)$ columns of $\tilde{\bf X}$
are non-zero (recall that, a constraint evaluates to $1$ if it is
false).

In the ultimate step of operation, $\tilde{\bf X}$ is pre-multiplied
by ${\bf A^T}$ and then added to ${\bf y}$ to obtain $\widehat{\bf X}$. In
other words, the columns of $\tilde{\bf X}$ are used as assignment
vectors to evaluate $\eL$ again. We use the fact that $\eL$ has an
UNSAT value greater than $d$ to infer that at least $d+1$ columns of
$\widehat{\bf X}$ have at least $d$ fraction of entries in $\widehat{\bf X}$
are $1$'s. And hence the lemma.  \qed \\


\noindent In what follows, we will establish that it is possible to
decode an assignment $\pi$ for $\eL$ from an assignment $\Pi$ of
$\eL^{\otimes 2}$ under the guarantee that the UNSAT($\eL^{\otimes
  2}$) exceeds a certain amount.


\begin{lemma}[Tensor Decoding Lemma]\label{decode}
  For every ${\bf X}$ such that the UNSAT of $\eL^{\otimes
    2}\left({\bf X}\right)$ is less than $d \cdot (d+1)$, there is a
  unique ${\bf x}$ such that the UNSAT$\left(\eL\right) = d$.
\end{lemma}
\noindent {\em Proof.} We give a proof by contradiction. Suppose that
the hypothesis is false. Say there is no unique ${\bf x}$ such that
${\bf x}^{\otimes 2} = {\bf X}$ and yet UNSAT($\eL^{\oplus 2}$) $\ < \
d \cdot(d+1)$, this contradicts Lemma \ref{decode1}.  And this
completes the lemma. \qed \\



\noindent A comment on the above lemma is due. Lemma \ref{decode}
implies that if a tensored system satisfies a certain fraction of
equations, then the tensored assignment is constructed from a unique
assignment of $\eL$.  Alternatively, rephrasing it in the language of
PCPs, if the verifier accepts a tensored with sufficiently high
probability, it necessarily means that the tensored proof is obtained
by tensoring a unique assignment to untensored system.

The key idea was that we set up ourselves such that we operate in the
{\em unique decodability regime} of the code generated by$A^{\otimes
  2}$. Specifically, we used that fact that the assignment fucntion
was symmetric and its UNSAT value is strictly less than $d \cdot (d +
1)$.


\begin{corollary}[Tensor Structure Lemma]
  For If a tensored verifier PCP system accepts a proof $\Pi$ with
  probability at least $(1 - (d^2 +d))$, then could construct ${\pi}$
  that would be accepted by the original verifier with probability at
  least $1 - d$.
\end{corollary}


%\eat{
We now extend the definition of tensoring to $k$ in a natural way.
\begin{definition}[$k$ Rounds of Tensoring]
  Given a linear system $\eL$, we define $\eL^{\otimes k}$ iteratively
  as $\eL^{\otimes k} = \eL^{\otimes k-1} \otimes \eL$.
\end{definition}





\begin{proposition}\label{nearly-there}
  For every ${\bf A,\ y}$ and ${\bf x} \ne \tilde{\bf x}$, ${\bf
    Ax\cdot y}^T + {\bf y}$({\bf Ax})$^T$ and ${\bf A} \tilde{\bf x}\cdot
    {\bf y}^T$+ {\bf y}({\bf A}$\tilde{\bf x}$)$^T$ differ in at least $y
  \cdot$ d($\A$) - d($\A$)$^2$ locations.
\end{proposition}
\noindent {\em Proof.}  We invoke Proposition \ref{different} between
the rows of ${\bf Ax\cdot y}^T$ and ${\bf A} \tilde{\bf x}\cdot {\bf
  y}^T$ (the first terms) to conclude they differ in no fewer than $y
\cdot$ d($\A$) locations. Along similar lines, we apply that very
argument over the columns of {\bf y}({\bf Ax})$^T$ and {\bf y}({\bf
  A}$\tilde{\bf x}$)$^T$ (the second terms) to conclude they differ
across $d($\A$)$ columns and each of them has a weight at least $y$.
To conclude the argument, we observe that the rows of first terms and
columns of second terms can share at most d($\A$)$^2$ entries. Thus,
the two entries ${\bf Ax\cdot y}^T + {\bf y}$({\bf Ax})$^T$ and ${\bf
  A} \tilde{\bf x}\cdot {\bf y}^T + {\bf y}$({\bf A}$\tilde{\bf
  x}$)$^T$ differ on at least $y \cdot$ d($\A$) - d($\A$)$^2$
entries. \qed


\begin{lemma}\label{towards}
  For every $\eL$, ${\bf x}$ such that $\eL({\bf x})$ is $d$ and $X
  = x^{\otimes 2}$, ${\bf x} \ne \tilde{\bf x}$ the Hamming weight of
\begin{align}\label{lower}
{\bf AXA}^T + {\bf y}\left({\bf A} \tilde{\bf x}\right)^T + \left({\bf A} \tilde{\bf x} + {\bf y} \right)\cdot {\bf y}^T
\end{align}
is no less than $d^2$ + $y \cdot$ d($\A$) - d($\A$)$^2$.
\end{lemma}
\noindent {\em Proof.} Because $\eL$({\bf x} has an UNSAT of $d$,
Lemma \ref{basic} allows us to infer that Equation \eqref{heavy} has a
Hamming weight of exactly $d^2$. Now observe that, the first two terms
of Equations \eqref{heavy} and \eqref{lower} remain unchanged. So to
prove the claim it suffices to focus of the penultimate terms of the
aforementioned equations. Recall Proposition \ref{nearly-there} to
establish that the Hamming difference between ${\bf Ax\cdot y}^T +
{\bf y}$({\bf Ax})$^T$ and ${\bf A} \tilde{\bf x}\cdot {\bf y}^T$+
{\bf y}({\bf A}$\tilde{\bf x}$)$^T$ is at least $y \cdot$ d($\A$) -
d($\A$)$^2$.  This lets us conclude that using a different ${\bf x}$
is at least ($y \cdot$ d($\A$) - d($\A$)$^2$)-far from $d^2$. Hence,
the weight of \eqref{lower} is either $d^2$ + $y \cdot$ d($\A$) -
d($\A$)$^2$ or $d^2$ - $y \cdot$ d($\A$) - d($\A$)$^2$..  At this
juncture, we invoke Lemma \ref{converse} to rule out the latter case.
Therefore, the Lemma is good. \qed



$\{0,1\}$ 
}


\eat{
\begin{lemma}\label{baby}
  For any linear system $\eL, \delta < 1 - d$ and every {\bf X} such
  that the Hamming weight of
  \[ {\bf AXA}^T + {\bf y}{\bf Ax}^T + {\bf Ax}{\bf y}^T + {\bf y}{\bf
    y}^T
\]
is strictly less than $d \cdot (d + \delta)$, then it follows that
the UNSAT of $\eL$({\bf x}) is strictly less than ($d +\delta$). Here,
{\bf x} denotes the diagonal vector of {\bf X}.

\end{lemma}
\noindent {\em Proof.} We prove the contrapositive. Suppose, UNSAT
$\eL$({\bf x}) is $d + \delta$. We invoke Corollary \ref{babydecode}
on {\bf x} with wt$\left({\bf Ax} + {\bf y}\right) = (d + \delta)$ to
conclude that the Hamming weight of {\bf AXA}$^T$ + {\bf y}{\bf
  Ax}$^T$ + {\bf Ax}{\bf y}$^T$ + {\bf y}{\bf y}$^T$ is at least $d
\cdot (d + \delta)$. This contradicts our Hypothesis and hence, the
lemma is valid. \qed


The above lemma leaves does not give any guarantees when the UNSAT is
in the range $\left(d^2, d^2 + d\right)$ and also, on the structure of
{\bf X}. We now address this by showing that if {\bf X} has an UNSAT
in that regime, then the Hamming distance between {\bf X} ${\bf
  x}^{\otimes 2}$ is at most $\delta$.  } 
