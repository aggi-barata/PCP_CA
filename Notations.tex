\section{The PCP verifier}




\eat{A PCP system has several parameters. Ideally, we would like all
parameters, except the completeness, to be as small as possible.

\begin{enumerate}
\item {\em Completeness(c).} The minimum probability with which
the prover can convince a verifier about a correct proof.
\item {\em Soundness (s).} The maximum probability with which a verifier
accepts an incorrect proof.
\item {\em Queries (q).} The number of queries made by the verifier
to the purported proof.
%\item {\em Size (m).} 
\item {\em Randomness (r).} The number of random bits used by the verifier.
A typical verifier uses $O(\log n)$ randomness. However, there are
several recent results which use only $(1 + o(1))\cdot\log n$ randomness.
\item {\em Alphabet ($\Sigma$).} The alphabet used in the new proof. The 
size of the alphabet is a lower bound on the soundness of a PCP. So, if
one seeks a low error, having a large alphabet is imminent.
\end{enumerate}
}
We denote by $PCP_{c,s}\left[r, q\right]_\Sigma$ the class of
languages that have a PCP verifier with completeness $c$, soundness
$s$, randomness $r$ and makes $q$ queries to the purported proof over
an alphabet $\Sigma$. It
is often useful to imagine all the parameters as a function of $n$.\\
\eat{
The PCP's with the projection property can also formulated a certain
CSP called the {\sc Label-cover} problem. An instance of a {\sc Label-cover}
is a bi-partitite graph whose edges are between questions of first prover and
those of the second prover. For every edge, there is an associated projection
that uniquely identifies the label of second vertex given the label of a vertex
from the first set. The goal is find a labelling that maximizes the number of
satisfied edges.  The following formalizes the notion we have been talking
in the prequel.

\begin{definition}[{\sc Label-cover}] An instance of {\sc Label-cover} 
contains a regular bi-partite multi-graph $G = (A, B, E)$ and two finite
sets $\Sigma_A$ and $\Sigma_B$, where $|\Sigma_A| \ge |\Sigma_B|$.
Every vertex in $A$ is supposed to get a label from $\Sigma_A$, and
every vertex in $B$ is supposed to get a label from $\Sigma_B$. For each
edge $e \in E$ there is a projection $\pi_e: \Sigma_A \rightarrow \Sigma_B$
which is a partial function.

Given a labeling to the vertices of the graph, that is, functions $\psi_A:A 
\rightarrow \Sigma_A$ and $\psi_B:B \rightarrow \Sigma_B$, an edge $e =(a,b)$
is said to be ``satisfied'' if $\pi_e(\psi_A(a)) = \psi_B(b)$ (if $\pi(\psi_A(a))$
is undefined; in which case the edge is 
deemed to be unsatisfied.) The goal is to find a labeling which 
maximizes the number of satisfied edges.
\end{definition} 

We say that $\gamma$ fraction of the edges are satisfiable if there exists a labeling
that satisfies $\gamma$ fraction of the edges. In the {\sc Label-cover} notation,
the size corresponds to the number of vertices $|A| + |B|$. The alphabet corresponds
to the larger set of labels $\Sigma_A$.

The {\sc Label-cover} seems to be extremely useful in establishing
inapproximability results. In particular, it is the starting point for
the famous H{\aa}stad's result (which also happens to be the backbone
of this work).

In the world of provers, the projection property is equivalent to a
$2$ prover $1$ round game and has the following correspondence with
the {\sc Label-cover}: The vertices in $A$ and $B$ are the set of
questions that can posed by the verifier to the first prover and the
second prover respectively. The set of labels corresponds to the
answers of the provers. So, upon receiving an answer from the first
prover, the verifier rejects the claim made by the provers or has
uniquely determined the answer to the second prover's question.

%\subsection{Error Correcting Codes}


We now introduce a couple of problems which are the mainstay of this
paper. First, we start with {\sc Max-$q$Lin}.  It is constraint
satisfaction problem over a set of variables. Every constraint is a
linear function over at most $q$ variables. The goal is find an
assignment to the variables which maximizes the fraction of satisfied
constraints. We define it in terms of its gap problem.

\begin{theorem}[{\sc GAP-$q$LIN}$_{\delta, \epsilon}$] Given a system
  of linear equations over $GF(2)$, each of which has $q$ variables,
  then for every $\delta, \epsilon > 0$, it is NP-hard to distinguish
  between the following:
\begin{itemize}
\item {\sf YES Case:} There is an assignment to the variables which
  satisfies $1-\delta$ fraction of the equations.
\item {\sf NO Case:} Every assignment satisfies only $\frac{1}{2} +
  \epsilon$ fraction of the equations.
\end{itemize}
\end{theorem}

It is well known that if one establishes that hardness of {\sc Gap-$q$Lin}$_{\delta, \epsilon}$, it implies that
approximating {\sc Max-$q$Lin} to $\delta/\epsilon$ is hard.}
